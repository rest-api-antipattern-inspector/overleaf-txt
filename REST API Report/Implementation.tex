\section{Implementation}

\subsection{Detection of Linguistic Antipatterns}

SARA (Semantic Analysis of RESTful APIs) is a Java program used for the detection of Linguistic Quality in URIs of REST APIs introduced by Palma et al. \cite{linguistic}. 

Our research will use SARA \cite{linguistic} for detecting the following linguistic antipatterns:
\begin{itemize}
\item Amorphous URIs
\item CRUDY URIs
\item Non-hierarchical Nodes
\item Inconsistent use of Singularised/Pluralised Nodes
\item Contextless Resource Names
\end{itemize}

%The following antipatterns can be detected with SARA but will be excluded from our research.
%\begin{itemize}
%\item Unversioned URIs - If an API in is not versioning their URIs there will be an effect on all the endpoints in that API. We think it is to difficult to draw a conclusion that there could be a relation with design quality.
%\item Inconsistent documentation - we want to focus on the structure of the URIs in question and not the documentation written about them.
%\item Less cohesive documentation -  see %above.
%\end{itemize}

SARA is built on JAVA and runs in the console. Each API is run separately and the result is stored in text files. Each text file consists of a list with URIs that are antipatterns and URI that passed the test.
The linguistic program has been further developed to get a more accurate result from finding context-less resources/URIs. Previously there was a limitation where nodes in a URI had words written together like `districtheat' where the following URI would become an antipattern: 

\vspace{2mm}
\texttt{/v3/{agreementId}/consumption/districtheat/data}

\vspace{2mm}
With our improvement, the program can detect words written together and treat them as separate nodes. If a word is not found in the dictionary then this code block is executed to check if the word might consist of more than one word. The two new words must make use of all the characters of the old word. See algorithm \ref{Find words written together algorithm} for the algorithm used for this improvement.

\begin{algorithm}
\caption{Find words written together}
\begin{algorithmic}
\State $word := Node.trim()$
\State $wordSingular := $""$ $
\If{$word.charAt(word.length() - 1) = $'s'$ $} 
    \State $wordSingular := word.substring(0, word.length() - 1)$
\EndIf
\If{$disco.frequency(word) > 0 \Or disco.frequency(wordSingular) > 0$}
    \State $uriNodes.add(word)$
\Else
\For{$i \In range(0, word.length())$}
    \If{$i = word.length() - 1$}
        \State $uriNodes.add(word)$
        \State \textbf{break}
    \EndIf
    
    \State $subWord1 := word.substring(0, i + 1)$
    \State $subWord2 := word.substring((i + 1), word.length())$
    
    \If{disco.frequency(subWord1.trim()) = 0}
        \State \textbf{continue}
    \EndIf
    
    \If{disco.frequency(subWord2.trim()) > 0}
        \State $uriNodes.add(subWord1)$
        \State $uriNodes.add(subWord2)$
        \State \textbf{break}
    \EndIf
\EndFor
\EndIf
\end{algorithmic}
\label{Find words written together algorithm}
\end{algorithm}

\clearpage

\subsection{Detection of REST Design Antipatterns}

To be able to detect REST design antipatterns in APIs, we have developed a program for automatically making API calls, analyzing the metadata and storing the results. 
The program for detecting REST design antipatterns is written in TypeScript using the Node.js platform. The program has a Command Line Interface (CLI), the user selects which APIs to run detection for and the results with meta data gets stored in a JSON file \footnote{\url{https://github.com/rest-api-antipattern-inspector/rest-api-antipattern-inspector}}. 

\subsection{Design Antipattern Detection Functions}

Algorithms \ref{Breaking Self-descriptiveness algorithm}, \ref{Forgetting Hypermedia algorithm}, \ref{Ignoring Caching algorithm}, \ref{Ignoring MIME Types algorithm}, \ref{Ignoring Status Code algorithm}, and \ref{Misusing Cookies algorithm} show the algorithms for the functions we have created to detect REST design antipatterns. Many of these functions use helper functions we have created in other parts of the source code, it returns the boolean value true if the antipattern is detected and false otherwise.

As shown in algorithm \ref{Breaking Self-descriptiveness algorithm}, the Breaking Self-descriptiveness antipattern gets detected if there are any non-standard headers in the request or response headers after an API call.

\begin{algorithm}
\caption{The detection of Breaking Self-descriptiveness.}
\begin{algorithmic}
\Function{$BSD$}{$requestHeaders, responseHeaders, nonStandardHeaders$}
\State $requestHeaderKeys := getKeys(requestHeaders)$
\State $responseHeaderKeys := getKeys(responseHeaders)$

\For{$headerKey \In requestHeaderKeys$}
    \If{$!getStandardRequestHeaders().includes(headerKey)$}
        \State $nonStandardHeaders.add(headerKey)$
    \EndIf
\EndFor

\For{$headerKey \In responseHeaderKeys$}
    \If{$!getStandardResponseHeaders().includes(headerKey)$}
        \State $nonStandardHeaders.add(headerKey)$
    \EndIf
\EndFor
\State \Return $!nonStandardHeaders.length \neq 0$
\EndFunction
\end{algorithmic}
\label{Breaking Self-descriptiveness algorithm}
\end{algorithm}

As shown in algorithm \ref{Forgetting Hypermedia algorithm}, the Forgetting Hypermedia antipattern gets detected if the response body to a GET request does not contain any link keys (link/links/href) or if a POST request does not contain that or a Location header.

\begin{algorithm}
\caption{The detection of Forgetting Hypermedia.}
\begin{algorithmic}
\Function{$FH$}{$body, httpMethod, responseHeaders$}
\State $bodyKeys := getAllKeys(body)$
\State \Return $( \newline 
\hspace*{3em} (httpMethod = $"GET"$ \And !containsLinks(bodyKeys)) \Or \newline 
\hspace*{3em} (httpMethod = $"POST"$ \And \newline 
\hspace*{4em} !containsHeader(responseHeaders, $"Location"$) \And \newline 
\hspace*{4em} !containsLinks(bodyKeys)) \newline 
\hspace*{2em})$
\EndFunction
\end{algorithmic}
\label{Forgetting Hypermedia algorithm}
\end{algorithm}

As shown in algorithm \ref{Ignoring Caching algorithm}, the Ignoring Caching antipattern gets detected for responses to GET requests if the response headers do not contain an ETag or if the request or response headers does not contain a Cache-Control header or if that is set to `no-cache' or `no-store'.

\begin{algorithm}
\caption{The detection of Ignoring Caching.}
\begin{algorithmic}
\Function{$IC$}{$httpMethod, requestHeaders, responseHeaders$}
\If{$httpMethod \neq $"GET"$ $}
\State \Return $false$
\EndIf
\State $clientCaching := getHeaderValue(requestHeaders, $"Cache-Control"$)$
\State $serverCaching := getHeaderValue(responseHeaders, $"Cache-Control"$)$
\State \Return $( \newline
\hspace*{3em} (!containsHeader(responseHeaders, $"Cache-Control"$) \Or \newline
\hspace*{4em} isNoCacheOrNoStore(clientCaching) \Or \newline
\hspace*{4em} isNoCacheOrNoStore(serverCaching)) \And \newline
\hspace*{3em} !containsHeader(responseHeaders, $"Etag"$) \newline
\hspace*{2em})$
\EndFunction
\end{algorithmic}
\label{Ignoring Caching algorithm}
\end{algorithm}

As shown in algorithm \ref{Ignoring MIME Types algorithm}, the Ignoring MIME Types gets detected if the response Content-Type header's value is not a standard mime type or if it is not among the accepted mime types requested.

\begin{algorithm}
\caption{The detection of Ignoring MIME Types.}
\begin{algorithmic}
\Function{$IMT$}{$requestHeaders, responseHeaders$}
\State $acceptedMIMETypes := getHeaderValues(requestHeaders, $"Accept"$)$
\State $contentType := getHeaderValue(responseHeaders, $"Content-Type"$)$
\State \Return $( \newline
\hspace*{3em} !contentType \Or \newline
\hspace*{3em} !isStandardMIMEType(contentType) \Or \newline
\hspace*{3em} !isAcceptedMIMEType(contentType, acceptedMIMETypes) \newline
\hspace*{2em})$
\EndFunction
\end{algorithmic}
\label{Ignoring MIME Types algorithm}
\end{algorithm}

As shown in algorithm \ref{Ignoring Status Code algorithm}, if the HTTP method used is GET, POST, PUT or DELETE, antipattern Ignoring Status Code gets detected if the combination of HTTP method, status code and status text in the response is not part of a list of standard combinations in an XML file used in the implementation by Palma et al.  \cite{linguistic}.

\begin{algorithm}
\caption{The detection of Ignoring Status Code.}
\begin{algorithmic}
\Function{$ISC$}{$\newline
\hspace*{1.4em} httpMethod, \newline
\hspace*{1.4em} statusCode, \newline
\hspace*{1.4em} statusText, \newline
\hspace*{1.4em} standardStatusCombos \newline
$}
\If{$httpMethod \neq (\newline
    \hspace*{3em} $"GET"$ \Or \newline
    \hspace*{3em} $"POST"$ \Or \newline
    \hspace*{3em} $"PUT"$ \Or \newline
    \hspace*{3em} $"DELETE"$ \newline
\hspace*{2em})$}
\State \Return $false$
\EndIf
\State \Return $(\newline
\hspace*{3em} standardStatusCombos.filter(\newline
\hspace*{4em} combo \ArrowFunction \newline
\hspace*{5em} combo.method = httpMethod \And \newline
\hspace*{5em} combo.code = statusCode \And \newline
\hspace*{5em} combo.text = statusText \newline
\hspace*{3em} ).length = 0 \newline
\hspace*{2em})$
\EndFunction
\end{algorithmic}
\label{Ignoring Status Code algorithm}
\end{algorithm}

As shown in algorithm \ref{Misusing Cookies algorithm}, the Misusing Cookies antipattern gets detected if the request or response headers contain any type of cookie header.

\begin{algorithm}
\caption{The detection of Misusing Cookies.}
\begin{algorithmic}
\Function{$MC$}{$requestHeaders, responseHeaders$}
\State \Return $(\newline
\hspace*{3em} containsCookieHeader(requestHeaders) \Or \newline 
\hspace*{3em} containsCookieHeader(responseHeaders) \newline
\hspace*{2em})$
\EndFunction
\end{algorithmic}
\label{Misusing Cookies algorithm}
\end{algorithm}

\clearpage

\subsection{Detection of Design Patterns}

After detecting design antipatterns and storing meta data about them, a function is used to determine and store information about patterns in our Node.js data-handler program. The patterns are set to true, meaning they are fulfilled, if the corresponding antipatterns are set to false, meaning no such antipattern was detected\footnote{\url{https://github.com/rest-api-antipattern-inspector/data-handler}}.

\subsection{Pairwise Phi Coefficient Calculations} \label{pairphico}

The scikit-learn library for Python includes a function for calculating the Phi Coefficient (or Matthews correlation coefficient, as it can also be called) \cite{scikitmcc}. We have used this for our pairwise Phi Coefficient calculations. 

As arguments for the function, two integer arrays are used. Both arrays contain one integer for every endpoint used. One array is for a design quality aspect (antipattern/pattern), the integers' values are 1 if the quality aspect has been detected in the corresponding endpoint and 0 if it has not. In the same way there is an array structured the same way for a linguistic quality aspect. The function then returns a result as a floating point between -1 and 1, more about that in the paragraph below. In this way, all design quality aspects are compared to all linguistic quality aspects.

The resulting Phi Coefficient floating point indicates the strength of the relation between the aspects. This number is between -1 and 1. A Phi Coefficient above 0 indicates a positive relation and one below 0 indicates a negative relation. 0-0.19 indicates a negligible positive relation, 0.2-0.29 a weak positive, 0.3-0.39 a moderate positive and above that is a strong or very strong positive and vice versa is the case for negative numbers indicating the strength of negative relations \cite{phico}. Osborn describes ranges from -0.3 to 0.3 as trivial \cite{Osborn}. 

\newpage