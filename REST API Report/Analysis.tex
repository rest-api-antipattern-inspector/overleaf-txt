\section{Analysis}

While answering the high-level research question  \textbf{RQ1} on \textit{Is there a relation between design- and linguistic quality in RESTful APIs?} \textcolor{blue}{Yes.} Based on Chi square test there is a relationship between design and linguistic quality in REST APIs. The Chi square test where a combination of antipatterns, patterns, and a mix of both. \textcolor{blue}{The p-value was very low indicating a significant correlation.}

While answering the first sub-research question \textbf{RQ1.1} on \textit{Is there a relation between design and linguistic antipatterns in RESTful APIs?} \textcolor{blue}{Yes.} according to the Phi coefficient pairwise tests between the design and linguistic antipatterns there is a moderate positive relationship between CRUDy URIs and Misusing Cookies. This is the strongest relation followed by a weak positive relationship for Contextless Resource Names vs Ignoring Status Code.

While answering the second sub-research question \textbf{RQ1.2} on \textit{Is there a relation between design patterns and linguistic antipatterns in RESTful APIs?} \textcolor{blue}{No.} We only define three patterns for REST design and two of these - Response Caching and  Entity Linking - are present in every endpoint. In other words there where no corresponding antipattern detected. This most likely impacted this result. The results indicate a weak negative relation between the linguistic pattern Tidy URIs and the design pattern Content Negotiation. Not much else can be said for this research question.

While answering the third sub-research question  \textbf{RQ1.3} on \textit{Is there a relation between design antipatterns and linguistic patterns in RESTful APIs?} \textcolor{blue}{Yes.} We can see that there is an overall negative relationship between these two. If there are design antipatterns linguistic pattern seem to be less common also. As in RQ1.2, a lot of the pairwise tests could not yield any definitive conclusion.

While answering the fourth sub-research question \textbf{RQ1.4} on \textit{Is there a relation between design- and linguistic patterns in RESTful APIs?} \textcolor{blue}{Yes.} Here we find a weak positive relationship in the pairwise phi coefficient tests between Amorphous URIs and Content Negotiation.

\newpage