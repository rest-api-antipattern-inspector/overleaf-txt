\section{Method}\label{sec:Method}

We have collected endpoints from ten different REST APIs and implemented the detection algorithms for linguistic and design quality based on the literature. 

We have created a Node.js program for making HTTP requests to the selected API endpoints and check for REST design antipatterns. For detecting linguistic patterns and antipatterns the Java program created by Palma et al. for their study of linguistic REST antipatterns \cite{linguistic} for checking for those types of patterns and antipatterns in the endpoints was used. 

To answer RQ1 we have performed a Chi-Square test on the design quality data \textit{vs.} the linguistic quality data, see table \ref{contingencytable} for more about the Chi-Square test. To answer the other research questions, we have performed pairwise Phi coefficient analyses.The list below describes the steps in our research method.

\begin{enumerate}
    \item Collect endpoints from 10 different APIs. Because of time constraint we have to limit the amount of APIs to include. Our supervisor recommenced using at least 10 different APIs. See \ref{APIsusedintheresearch} for full list.
    \item Define rules for linguistic and design patterns/antipatterns. We need to define rules for implementing our detection functions. The rules will be based on earlier literature on the subject \cite{design} \cite{linguistic}. See subsections \ref{desidnAntipatterns}, \ref{designPatterns}, \ref{linguisticAntipatterns} and \ref{linguisticPatterns} for all rules defined.
    \item Write functionality in Node.js for detecting design quality in API endpoints and storing the results. We use the Node.js platform for making our HTTP request calls from.
    \item Prepare the Java program SARA for detecting linguistic quality in our collected endpoints. The JAVA program SARA has been developed and used in earlier research for detecting linguistic quality \cite{linguistic}.
        \item Run programs with our collected endpoints to detect design and linguistic quality as well as store the result. The result will be used for statistical analysis to determine the relation between linguistic and design quality. More details about this in the implementation chapter. 
    \item For answering RQ1: What is the relationship between design and linguistic quality? Perform Chi-Square analysis on the design quality data vs linguistic quality data. See subsection \ref{statistics} for the motivation for this statistical analysis. 
        \item For answering RQ1.1 - RQ1.4: Run pairwise phi coefficient tests between each combination of pattern and antipattern. See subsection \ref{statistics} for the motivation for this type of statistical analysis.
\end{enumerate}

\subsection{APIs Used for the Study} \label{selectedAPIs}

We selected ten REST APIs for this study. We mosly based our selection of these on the APIs used in previous research by Palma et al. \cite{design}\cite{linguistic}. For some APIs used by Palma et al., we selected other similar APIs as alternatives. The APIs we selected are listed below with links to their documentations in footnotes and with our motivations for selecting them. 

\begin{itemize}
\label{APIsusedintheresearch}
    \item \textbf{Bitly\footnote{\url{https://dev.bitly.com/}} - version 3:} used by Palma et al. \cite{linguistic}.
    \item \textbf{Disqus\footnote{\url{https://disqus.com/api/docs/}} - version 3:} Selected because Disqus  is  a  widely used service.
    \item \textbf{Facebook\footnote{\url{https://developers.facebook.com/docs/apis-and-sdks/}} - version 7.0:} used by Palma et al. \cite{linguistic}.
    \item \textbf{GitHub\footnote{\url{https://developer.github.com/v3/}} - version 3:} one API used by Palma et al. was Ohloh \cite{linguistic}. The Ohloh website does not exist as Ohloh anymore \cite{ohloh}, since Ohloh provided services to support open source software \cite{ohloh} we decided to use GitHub as an alternative since GitHub also provides services for open source software and is, at the time of writing, very well known and widely used. 
    \item \textbf{Imgur\footnote{\url{https://apidocs.imgur.com/?version=latest}} - version 3:} one API used by Palma et al. was Instagram \cite{linguistic}, since Instagram has been acquired by Facebook and we are using Facebook, we decided to pick an alternative to Instagram to make our list of APIs studied more diversified. As an alternative to Instagram, we selected Imgur, which is also a widely used image sharing service.
    \item \textbf{Nasa\footnote{\url{https://ssd-api.jpl.nasa.gov/doc/index.php}}:} Selected because Nasa is a widely known organization
    \item \textbf{Spotify\footnote{\url{https://developer.spotify.com/documentation/}} - version 1:} one API used by Palma et al. was MusicGraph \cite{linguistic}. MusicGraph is no longer available on RapidAPI \cite{musicrapid}. As a replacement for MusicGraph, we selected Spotify, which is a widely used and well known music service. 
    \item \textbf{StackExchange\footnote{\url{https://api.stackexchange.com/}} - version 2.2:} used by Palma et al. \cite{linguistic}.
    \item \textbf{Twitter\footnote{\url{https://developer.twitter.com/en}} - version 1.1:} used by Palma et al. \cite{linguistic}.
    \item \textbf{Vimeo\footnote{\url{https://developer.vimeo.com/}} - version 3:} one API used by Palma et al. \cite{linguistic} was YouTube, since we are excluding Google APIs, we selected Vimeo as an alternative, Vimeo is also a widely used video sharing platform. 
\end{itemize}

\subsection{Statistical analyses for determining relations} \label{statistics}

We have performed two types of statistical analyses to determine the relations between linguistic and design quality to answer our research questions in subsection \ref{research_questions}, these two types of statistical analyses are Chi-Square and Phi Coefficient. Osborn describes Chi-Square as useful for determining whether or not two variables are statistically associated \cite{Osborn}. However, Osborn also claims that Chi-Square does not show the strength of association and recommends Phi Coefficent for determining that \cite{Osborn}. The American Psychological Association also describes Phi Coefficient as useful for determining statistical association of binary variables \cite{apa}. 

Our implementation of Chi-Square is discussed further in subsection \ref{overallstats}. Our implementation of Phi Coefficient is discussed further in subsection \ref{pairphico}.

\subsection{Reliability and Validity} \label{Reliability and Validity}

To increase the reliability, our study includes several appendixes, for example, links to repositories/downloads of the source code used (with documentation for how to use it), as well as a document of all the API endpoints used. This will make it possible for others to recreate and inspect the research conducted. 

We have written unit tests for the antipattern detection methods in the Node.js program for detecting REST design antipatterns. These will be called with mocked network responses that contain antipatterns to verify that our implementation is correct. They will also be provided with responses that do not contain antipatterns to avoid false positives.

The result from the JAVA program SARA which measures the linguistic quality in APIs needs to be verified. This will be done by verifying the correctness of a sample of endpoints tested through the program and see if those endpoints that had a detected antipattern was not a false positive and vice versa.

One reliability risk is that the API versions used might become outdated and replaced with new versions. The API versions used have therefore been listed to let future researchers that might want to replicate our findings know which API versions were used. Hopefully the API versions we used will still be supported.

Another reliability risk is that the APIs could fix some errors detected which could produce different results if the detection programs used in this study are used again in the future. However, large changes within an API version should be unlikely. 
The biggest challenge will be to achieve a high validity. Time is a limited resource, there are limits to the amount of APIs that can be inspected and how many endpoints can be inspected in those. The  selection of APIs will be based on previous research by Palma et al. \cite{linguistic} and Google APIs will be excluded from them. From those APIs endpoints will be prioritized that have unique qualities and are prominent in the documentation. 

Another threat to validity is that our findings might become less relevant if the problems found are fixed in later versions. However, even if the results would become significantly different in future versions, the result of this study would still provide a snapshot of current conditions. 

The APIs selected for this study are listed and motivated in subsection \ref{selectedAPIs}. We mostly based the selection of these APIs on the APIs used in previous research by Palma et al. \cite{design}\cite{linguistic}, as discussed, for some APIs used in the previous research, we selected similar APIs as alternatives. We selected some of the APIs used by Palma et al. to make our results comparable. We also strove to select APIs that are widely known and we also strove to have a variety in the selection. With a variety of widely known APIs we hoped to achieve a high validity.

\newpage